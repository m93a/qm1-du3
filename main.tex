% !TEX program = xelatex

\documentclass{article}
\usepackage[utf8]{inputenc}
\usepackage[T1]{fontenc}
\usepackage{lmodern}
\usepackage[czech]{babel}
\usepackage[top = 1.5cm, bottom = 1.5cm, left = 2cm, right = 2cm]{geometry}
\usepackage{amsmath}
\usepackage{amssymb}
\usepackage{mathtools}
\usepackage{letltxmacro}
\usepackage{wrapfig}
\usepackage{bbm}


% import obrázků z Inkscape
% podle návodu na https://castel.dev/post/lecture-notes-2/
\usepackage{import}
\usepackage{xifthen}
\usepackage{pdfpages}
\usepackage{transparent}

\newcommand{\incfig}[1]{%
    \def\svgwidth{\columnwidth}
    \import{./}{#1.pdf_tex}
}


\LetLtxMacro{\oldhbar}{\hbar}
\renewcommand*{\hbar}{{\mathpalette\hbaraux\relax\mathrm{h}}}
\newcommand*{\hbaraux}[2]{\sbox0{\mathsurround=0pt$#1\mathchar'26$}\mkern-1mu\lower.07\ht0\box0\mkern-8mu}

\def\ph{\phantom}
\def\vph{\vphantom}
\def\hph{\hphantom}
\def\rzw{\mathrlap}
\def\lzw{\mathllap}
\def\czw{\mathclap}

\newcommand{\comm}[2]{\left[ #1, #2 \right]}
\newcommand{\const}[1]{\text{#1}}
\newcommand{\norm}[1]{\left\lVert#1\right\rVert}
\newcommand{\innerprod}[2]{\big< #1, #2 \big>}
\newcommand{\mean}[1]{\big< #1 \big>}
\renewcommand{\d}[1]{\;\const{d}#1}
\newcommand{\dd}[2]{\frac{\const{d} #1}{\const{d} #2} \;}
\newcommand{\pd}[2]{\frac{\partial  #1}{\partial  #2} \;}
\newcommand{\e}[1]{\const{e}^{#1}}
\renewcommand{\i}{\const{i}}
\newcommand{\bhat}[1]{\hat{\bm{#1}}}
\newcommand{\vechat}[1]{\hat{\vec{#1}}}
\renewcommand{\dot}[1]{\accentset{\bullet}{#1}}
\newcommand{\Tr}{\operatorname{Tr}}
\newcommand{\bra}[1]{\left< #1 \right|}
\newcommand{\ket}[1]{\left| #1 \right>}
\newcommand{\braket}[2]{\left< #1 \middle| #2 \right>}
\newcommand{\f}{\varphi}
\newcommand{\Parity}{\hat{\mathcal{P}}}
\newcommand{\R}{\mathbb{R}}
\newcommand{\C}{\mathbb{C}}

\newcommand{\mat}[1]{
    \begin{pmatrix}
        #1
    \end{pmatrix}
}

\newcommand{\mata}[2]{
    \left(
    \begin{array}{@{}#1@{}}
        #2
    \end{array}
    \right)
}

\newcommand{\smat}[2][1]{
    \scalebox{#1}{$\mat{#2}$}
}

\begin{document}

\section*{Úkol 3: Provázání dvou qubitů}
\textbf{Autor: Michal Grňo}

\subsection*{Zadání}
Máme dvouhladinový systém odpovídající qubitu $\mathcal H = \C^2 = \operatorname{span} \{ \ket{0}, \ket{1} \}$. Na tomto prostoru máme operátory
\begin{align*}
    \hat I & = \smat[0.9]{1 \\& 1} \: , &
    \hat h &= \smat[0.9]{0 \\& \epsilon} \: , &
    \hat \sigma_x &= \smat[0.9]{ & 1 \\ 1 } \: , &
    \hat \sigma_y &= \smat[0.9]{ &\!\!-\i \\ \i } \: .
\end{align*}
Bude nás zajímat systém dvou qubitů, mezi kterými nejprve neprobíhá žádná interakce; hamiltonián systému je
\begin{align*}
    \hat H_0 = \hat h \otimes \hat I + \hat I \otimes \hat h \: ,
\end{align*}
můžeme ale zapnout interakci s celkovým hamiltoniánem
\begin{align*}
    \hat H &= \hat H_0 + \hat V \: , &
    \hat V = \hbar \omega (\hat \sigma_x \otimes \hat \sigma_y - \hat \sigma_y \otimes \hat \sigma_x) \: .
\end{align*}
Budeme uvažovat počáteční stav $\ket \psi = \ket 1 \otimes \ket 0$. Jak se bude tento stav vyvíjet při vypnuté interakci? Jak při zapnuté? Dále budeme uvažovat stav $\ket{\psi(T)}$ vytvořený tak, že jsme pro stav $\ket \psi$ na čas $T=\frac{\pi}{8\omega}$ zapneme interakci. Ukažte, že je tento stav provázaný. Jaké jsou pravděpodobnosti výsledků měření $\hat h \otimes \hat I$ a následně $\hat I \otimes \hat h$?

\subsection*{Řešení}
Na součinovém prostoru $\mathcal H \otimes \mathcal H$ budeme používat Kroneckerův součin matic:
\begin{align*}
    \hat H_0
    = \hat h \otimes \hat I + \hat I \otimes \hat h
    = \smat[0.9]{0 \\& \epsilon} \otimes \smat[0.9]{1\\&1}
    + \smat[0.9]{1\\&1} \otimes \smat[0.9]{0 \\& \epsilon}
    = \smat[0.9]{0\\&0\\&&\epsilon\\&&&\epsilon}
    + \smat[0.9]{0\\&\epsilon\\&&0\\&&&\epsilon}
    = \smat[0.9]{0\\&\epsilon\\&&\epsilon\\&&&2\epsilon} \: .
\end{align*}
\begin{align*}
    \hat V
    = \hbar \omega (\hat \sigma_x \otimes \hat \sigma_y - \hat \sigma_y \otimes \hat \sigma_x)
    = \hbar \omega \smat[0.9]{&&& \!-\i \\&& \i \\& \!-\i \\ \i}
    - \hbar \omega \smat[0.9]{&&& \!-\i \\&& \!-\i \\& \i \\ \i}
    = \smat[0.9]{0 \\& 0 & 2\i\hbar\omega \\& \!-2\i\hbar\omega & 0 \\&&& 0} \: ,
\end{align*}
\begin{align*}
    \hat H = \smat[0.9]{0 \\& \epsilon & 2\i\hbar\omega \\& \!-2\i\hbar\omega & \epsilon \\&&& 2\epsilon} \: .
\end{align*}
Matice hamiltoniánu $\hat H$ je diagonalizovatelná a blokově diagonální, její „prostřední“ blok diagonalizujeme snadno:
\begin{align*}
    \smat[0.9]{\epsilon & 2\i\hbar\omega \\ \!-2\i\hbar\omega & \epsilon}
    \quad=\quad
    \frac{1}{\sqrt{2}} \smat[0.9]{\!-\i & \i \\ 1 & 1} \quad
    \smat[0.9]{\epsilon - 2\hbar\omega \\& \epsilon + 2\hbar\omega} \quad
    \frac{1}{\sqrt{2}} \smat[0.9]{\i & 1 \\ \!-\i & 1}
\end{align*}
Evoluční matice pro vypnutou, resp. zapnutou interakci tedy budou
\begin{align*}
    \hat U_0(t) = \exp(\, -\i \, \frac{t}{\hbar} \, \hat H_0 \,)
    &= \smat{1 \\& \e{\displaystyle -\i \varepsilon t} \\&& \e{\displaystyle -\i \varepsilon t} \\&&& \e{\displaystyle -\i 2\varepsilon t} } \: , &
    \varepsilon &\coloneqq \frac{\epsilon}{\hbar} \: .
\end{align*}
\begin{gather*}
    \hat U(t) = \exp(\, -\i \, \frac{t}{\hbar} \, \hat H \, )
    = \frac{1}{\sqrt{2}} \smat[0.9]{1 \\& \!-\i & \i \\& 1 & 1 \\&&& 1} \quad
    \smat{1 \\& \e{\displaystyle -\i (\varepsilon - 2\omega)t} \\&& \e{\displaystyle -\i (\varepsilon + 2\omega)t} \\&&& \e{\displaystyle -\i 2\varepsilon t}} \quad
    \frac{1}{\sqrt{2}} \smat[0.9]{1 \\& \i & 1 \\& \!-\i & 1 \\&&& 1}
\\
    \hat U(t) = \smat{
        \;1
        \\& \e{\displaystyle -\i\varepsilon t} \cos 2\omega t & -\e{\displaystyle -\i\varepsilon t} \sin 2\omega t
        \\& \e{\displaystyle -\i\varepsilon t} \sin 2\omega t & \ph{-}\e{\displaystyle -\i\varepsilon t} \cos 2\omega t
        \\&&& \e{\displaystyle -\i 2\varepsilon t}\,
    }
\end{gather*}
Nyní už se můžeme podívat na chování samotného stavu $\ket\psi$. V naší součinové bázi  ho představuje vektor
\begin{align*}
    \ket\psi
    = \ket 1 \otimes \ket 0
    = \smat[0.9]{0\\1} \otimes \smat[0.9]{1\\0}
    = \smat[0.9]{0\\0\\1\\0}
    \: .
\end{align*}
Protože je $\ket\psi$ stacionárním stavem bezinterakčního hamiltoniánu, jeho vývoj je triviální:
\begin{align*}
    \ket{ \psi_0(t) } = \hat U_0(t) \ket\psi = \e{\displaystyle -\i\varepsilon t} \ket\psi
    \: .
\end{align*}
Oproti tomu se zapnutou interakcí je vývoj složitější:
\begin{align*}
    \ket{\psi(t)} = \hat U(t) \ket\psi
    = \e{\displaystyle -\i\varepsilon t} \, \smat[0.9]{0\\\!-\sin 2\omega t\\\ph{\!-}\cos 2\omega t \\0}
    = \e{\displaystyle -\i\varepsilon t} \big(
        -\sin 2\omega t \, \ket{01} +
        \cos 2\omega t \, \ket{10}
    \big) \: .
\end{align*}
Specificky nás zajímá, co se stane, když interakci spustíme na $T=\frac{\pi}{8\omega}$ a poté ji vypneme. Potom bude stav
\begin{align*}
    \ket{\psi(T)} = \frac{ \e{\displaystyle -\i \varepsilon T} }{\sqrt{2}} \,
    \smat[0.9]{0 \\ \!-1 \\ 1 \\ 0}
    \: .
\end{align*}
To je (až na fázi) Bellův stav $\Psi^-$, tedy jeden z maximálně provázaných stavů dvou qubitů. Že se skutečně jedná o provázaný stav můžeme i dokázat. Provázaný stav je z definice takový stav složeného kvantového systému, který není tenzorovým součinem stavů jednotlivých podsystémů. Předpokládejme tedy, že náš stav \textit{je} součinovým stavem:
\begin{align*}
    \smat[0.9]{a\\b} \otimes \smat[0.9]{c\\d}
    = \smat[0.9]{ac\\ad\\bc\\bd}
    = \smat[0.9]{0 \\ \!-1 \\ 1 \\ 0}
\end{align*}
Z prostředních řádků $ad=-1$ a $bc=1$ dostáváme podmínku, že žádná z proměnných $a,b,c,d$ není nula. To je ovšem ve sporu s krajními řádky $ac=0$ a $bd=0$. Tudíž jsme dokázali, že stav $\ket{\psi(T)}$ musí být provázaným stavem.

\bigskip

Nakonec nás zajímají výsledky měření energie na jednotlivých qubitech. K tomu využijeme zhuštěnou notaci, kde redukci stavu $A$ na stav $B$ po naměření hodnoty $c$ budeme značit $A \stackrel{c}{\mapsto} B$.
\begin{align*}
    \frac{1}{\sqrt{2}} \, \big(-\ket{01} + \ket{10} \big)
    & \quad\stackrel{0}{\mapsto}\quad
    \frac{-1}{\sqrt{2}} \, \ket{01}
    \quad\stackrel{0}{\mapsto}\quad 0
    \\[5pt]
    \frac{1}{\sqrt{2}} \, \big(-\ket{01} + \ket{10} \big)
    & \quad\stackrel{0}{\mapsto}\quad
    \frac{-1}{\sqrt{2}} \, \ket{01}
    \quad\stackrel{\displaystyle \epsilon}{\mapsto}\quad
    \frac{-1}{\sqrt{2}} \, \ket{01}
    \\[5pt]
    \frac{1}{\sqrt{2}} \, \big(-\ket{01} + \ket{10} \big)
    & \quad\stackrel{\displaystyle \epsilon}{\mapsto}\quad
    \frac{1}{\sqrt{2}} \, \ket{10}
    \quad\stackrel{0}{\mapsto}\quad
    \frac{1}{\sqrt{2}} \, \ket{10}
    \\[5pt]
    \frac{1}{\sqrt{2}} \, \big(-\ket{01} + \ket{10} \big)
    & \quad\stackrel{\displaystyle \epsilon}{\mapsto}\quad
    \frac{1}{\sqrt{2}} \, \ket{10}
    \quad\stackrel{\displaystyle \epsilon}{\mapsto}\quad 0
\end{align*}
\begin{wrapfigure}{r}{5cm}
    \vspace{-1\baselineskip}
    \centering
    \includegraphics[width=4cm]{kocka.png}
    \caption{Kočka ve stavu $\ket{x+}$}
\end{wrapfigure}
Vidíme, že jediné energie, které můžeme naměřit, jsou $(\epsilon, 0)$ a $(0, \epsilon)$, obě s pravděpodobností $50\,\%$.



\subsubsection*{Závěr}
Jako omluvu za pozdní odevzdání přijměte obrázek.




\end{document}
